\documentclass[border=10pt]{standalone}

\usepackage{luatexja}

\usepackage{tikz}

% From [this answer by @JasperHabicht](https://tex.stackexchange.com/a/730333/64441).
\pgfkeys{
  /tikz/pics/shogi grid/.style={
    code={
      \draw (0, 0) grid[step = 1] 
        (\shogiBoardDimension, \shogiBoardDimension * \shogiHeightMultiplier); 
    },
  },
  /tikz/pics/shogi piece/.style={
    code={
      \pgfkeys{/tikz/shift={(0.5, 0.5)}, /phili/shogiban/piece/.cd, #1}
      \draw[
        draw   = \shogiPieceColor,
        scale  = \shogiPieceScale,
        rotate = \shogiPieceRotate,
      ]
        (0, 0.35) --
        (-0.28, 0.23) --
        (-0.3, -0.3) --
        (0.3, -0.3) --
        (0.28, 0.23) --
        cycle;
      \node[
        font   = \fontsize{7.5 pt}{7.5 pt}\selectfont,
        color  = \shogiPieceColor,
        scale  = \shogiPieceScale,
      ] at (0, 0) {\rotatebox{\shogiPieceRotate}{\shogiPieceText}};
    }
  },
  /phili/shogiban/.cd, 
    board dimension/.store in   = \shogiBoardDimension,
    board dimension             = 5 cm,
    height multiplier/.store in = \shogiHeightMultiplier,
    height multiplier           = 1.2,
    initial scale/.style        = {
      /tikz/x                   = \shogiBoardDimension / 9 cm * 1 cm,
      /tikz/y                   = \shogiBoardDimension * \shogiHeightMultiplier / 9 cm * 1 cm
    },
    piece/color/.store in       = \shogiPieceColor, 
    piece/color                 = black,
    piece/text/.store in        = \shogiPieceText,
    piece/text                  = A,
    piece/scale/.store in       = \shogiPieceScale,
    piece/scale                 = 1.0,
    piece/rotate/.store in      = \shogiPieceRotate,
    piece/rotate                = 0,
}

\newenvironment{shogiban}[1][]{%
  \begin{tikzpicture}[/phili/shogiban/.cd, #1, initial scale]
    \pic{shogi grid};
}{
  \end{tikzpicture}%
}

\begin{document}
  \begin{shogiban}[board dimension = 5 cm]
    \pic at (0, 0) {shogi piece = {text = 香}}; % Lance
    \pic at (1, 0) {shogi piece = {text = 桂}}; % Knight
    \pic at (2, 0) {shogi piece = {text = 銀}}; % Silver
    \pic at (3, 0) {shogi piece = {text = 金}}; % Gold
    \pic at (4, 0) {shogi piece = {text = 玉, scale = 1.1}}; % King (sente)
    \pic at (5, 0) {shogi piece = {text = 金}}; % Gold
    \pic at (6, 0) {shogi piece = {text = 銀}}; % Silver
    \pic at (7, 0) {shogi piece = {text = 桂}}; % Knight
    \pic at (8, 0) {shogi piece = {text = 香}}; % Lance

    \pic at (1, 1) {shogi piece = {text = 角}}; % Bishop
    \pic at (7, 1) {shogi piece = {text = 飛}}; % Bishop

    \foreach \x in {0, ..., 8} {
      \pic at (\x, 2) {shogi piece = {text = 歩}}; % Pawn  
    }

    \pic at (1, 3) {shogi piece = {color = red, text = 馬}}; % Promoted Bishop (Dragon Horse)
    \pic at (7, 3) {shogi piece = {color = red, text = 龍}}; % Promoted Rook (Dragon King)
    \pic at (3, 3) {shogi piece = {color = red, text = 全}}; % Promoted Silver (Gold)
    \pic at (4, 3) {shogi piece = {color = red, text = 今}}; % Promoted Knight
    \pic at (5, 3) {shogi piece = {color = red, text = 杏}}; % Promoted Lance
    \pic at (4, 4) {shogi piece = {color = red, text = と}}; % Promoted Pawn

    \pic at (4, 8) {shogi piece = {text = 王, scale = 1.1, rotate = 180}}; % King (gote)
  \end{shogiban}
\end{document}