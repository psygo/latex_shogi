\documentclass[12pt]{book}

% ~A6 (+ 5mm min left and right offset)
\usepackage[%
  paperheight   = 148mm, 
  paperwidth    = 105mm,
  bindingoffset = 5mm,
  left          = 5mm,
  right         = 8mm,
  top           = 15.2mm,
  bottom        = 13.2mm,
  footskip      = 6.35mm,
]{geometry}

\usepackage{luatexja}

\usepackage{tikz}

\pgfkeys{%
  /phili/shogiban/.cd, 
    board dimension/.store in   = \boardDimension,
    board dimension             = 5 cm,
    height multiplier/.store in = \heightMultiplier,
    height multiplier           = 1.2,
}

\newcommand{\calculateDimensions}{
  \pgfmathsetmacro{\width}{\boardDimension}
  \pgfmathsetmacro{\height}{\boardDimension * \heightMultiplier}
}

\newcommand{\calculateSteps}{
  \pgfmathsetmacro{\stepX}{\width / 9 cm}
  \pgfmathsetmacro{\stepY}{\height / 9 cm}
}
    
\newcommand{\shogiGrid}[1][]{
  \draw[xstep = \stepX, ystep = \stepY]
    (0, 0) grid
    (\boardDimension, \boardDimension * \heightMultiplier);
}

\newenvironment{shogiban}[1][]{
  \pgfkeys{/phili/shogiban/.cd, #1}

  \begin{tikzpicture}
    \calculateDimensions
    \calculateSteps

    \shogiGrid
}{
  \end{tikzpicture}
}

\pgfkeys{%
  /phili/piece/.cd, 
    x/.store in     = \pieceX,
    x               = 0,
    y/.store in     = \pieceY,
    y               = 0,
    text/.store in  = \pieceText,
    text            = A,
    color/.store in = \pieceColor,
    text            = black,
}

\newcommand{\piece}[1][]{
  \pgfkeys{/phili/piece/.cd, #1}
  
  \pgfmathsetmacro{\pieceCorrectedX}{(\pieceX + 0.5) * \stepX}
  \pgfmathsetmacro{\pieceCorrectedY}{(\pieceY + 0.5) * \stepY}

  \pgfmathsetmacro{\pieceBaseStartX}{(\pieceX + 0.2) * \stepX}
  \pgfmathsetmacro{\pieceBaseStartY}{(\pieceY + 0.2) * \stepY}
  \pgfmathsetmacro{\pieceBaseEndX}{(\pieceX + 0.8) * \stepX}
  \pgfmathsetmacro{\pieceBaseEndY}{(\pieceY + 0.2) * \stepY}
  \pgfmathsetmacro{\pieceTopLeftX}{(\pieceX + 0.22) * \stepX}
  \pgfmathsetmacro{\pieceTopLeftY}{(\pieceY + 0.73) * \stepY}
  \pgfmathsetmacro{\pieceTopRightX}{(\pieceX + 0.78) * \stepX}
  \pgfmathsetmacro{\pieceTopRightY}{(\pieceY + 0.73) * \stepY}
  \pgfmathsetmacro{\pieceTopX}{(\pieceX + 0.5) * \stepX}
  \pgfmathsetmacro{\pieceTopY}{(\pieceY + 0.85) * \stepY}

  \draw [draw = \pieceColor]
    (\pieceTopX, \pieceTopY) --
    (\pieceTopLeftX, \pieceTopLeftY) --
    (\pieceBaseStartX, \pieceBaseStartY) --
    (\pieceBaseEndX, \pieceBaseEndY) --
    (\pieceTopRightX, \pieceTopRightY) --
    (\pieceTopX, \pieceTopY);

  \draw (\pieceCorrectedX, \pieceCorrectedY) 
    node[
      font  = \fontsize{7.5pt}{7.5pt}\selectfont,
      color = \pieceColor,
    ] {\pieceText};
}

\begin{document}
  \begin{shogiban}[board dimension = 5cm]
    % Back Line
    \piece[x = 0, y = 0, text = 香, color = black] % Lance
    \piece[x = 1, y = 0, text = 桂, color = black] % Knight
    \piece[x = 2, y = 0, text = 銀, color = black] % Silver
    \piece[x = 3, y = 0, text = 金, color = black] % Gold
    \piece[x = 4, y = 0, text = 玉, color = black] % King (sente)
    \piece[x = 5, y = 0, text = 金, color = black] % Gold
    \piece[x = 6, y = 0, text = 銀, color = black] % Silver
    \piece[x = 7, y = 0, text = 桂, color = black] % Knight
    \piece[x = 8, y = 0, text = 香, color = black] % Lance
    % Middle Line
    \piece[x = 1, y = 1, text = 角, color = black] % Bishop
    \piece[x = 7, y = 1, text = 飛, color = black] % Bishop
    % Front Line
    \piece[x = 0, y = 2, text = 歩, color = black] % Bishop
    \piece[x = 1, y = 2, text = 歩, color = black] % Bishop
    \piece[x = 2, y = 2, text = 歩, color = black] % Bishop
    \piece[x = 3, y = 2, text = 歩, color = black] % Bishop
    \piece[x = 4, y = 2, text = 歩, color = black] % Bishop
    \piece[x = 5, y = 2, text = 歩, color = black] % Bishop
    \piece[x = 6, y = 2, text = 歩, color = black] % Bishop
    \piece[x = 7, y = 2, text = 歩, color = black] % Bishop
    \piece[x = 8, y = 2, text = 歩, color = black] % Bishop
    % Promoted Pieces
    \piece[x = 1, y = 3, text = 馬, color = red] % Promoted Bishop (Dragon Horse)
    \piece[x = 7, y = 3, text = 龍, color = red] % Promoted Rook (Dragon King)
    \piece[x = 3, y = 3, text = 全, color = red] % Promoted Silver (Gold)
    \piece[x = 4, y = 3, text = 今, color = red] % Promoted Knight
    \piece[x = 5, y = 3, text = 杏, color = red] % Promoted Lance
    \piece[x = 4, y = 4, text = と, color = red] % Promoted Pawn
  \end{shogiban}
\end{document}
  \end{shogiban}
\end{document}